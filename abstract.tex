% Francais

% • le contexte,
En 2018, Le Professeur Yves Chevallier a imaginé un nouveau format de sérialisation proche de YAML et JSON nommé UON.

UON vise à rassembler toutes les caractéristiques utlies de des formats de sérialisation les plus utilisés (XML, YAML et JSON) en un seul format qui les englobe et le rend adapté à la communication m2m, des dispositifs embarqués de faible puissance aux plateformes haut de gamme basées sur le cloud.

Il offre des fonctionnalités supplémentaires utiles pour accroître l'interopérabilité entre différents types d'appareils. Ses principaux avantages par rapport à d'autres langages similaires sont les suivants :

%Validation schema
%Binary payload
%Canonical form
%Physical quantities and units
%References
%Standard API
%Type properties
%User types

%Schéma de validation
%Charge utile binaire
%Forme canonique
%Quantités et unités physiques
%Références
%API standard
%Propriétés des types
%Types d'utilisateurs

\begin{itemize}
    \item Typage fort
    \item Schéma de validation
    \item Forme canonique
    \item Quantités et unités physiques
    \item Références
    \item API standard
    \item Propriétés des types
    \item Types d'utilisateurs
\end{itemize}

% • la problématique,
Pour être utilisable, ce format de sérialisation devrait disposer de son propre support de langage afin qu'il soit utilisé par un plus grand nombre.

% • les objectifs du travail,
% editeur de texte ou de code ?
Ce Travail de bachelor à l'objectif de répondre à ce besoin en permettant l'utilisation du langage UON dans un éditeur de code.
Sa syntaxe devra être pris en charge à terme par les pincipaux éditeurs de code. Nous nous focaliserons ici, uniquement sur l'éditeur VS Code.
Le support de langage consistera donc à fournir à l'utilisateur, les outils pour lui permettre la rédaction la plus optimale d'un fichier UON.

% • la méthodologie utilisée,
Le travail s'est effectué de manière agile tout en essaynt de respecter au mieux le planning.
Des discussions avec l'ensignant on eu lieu chaque semaines pour s'assurer de la bonne avancée du projet.

Les trois premières semaines ont été majoritairement consacré à se rensigner.
Un PoC à aussi vite été realisé pour s'assurer de la faisabilité de certaines fonctionnalités.

% • les principaux résultats de la recherche ou, si confidentiel, au moins la nature
%   des résultats,
Au terme de ce projet, les points attendues du cahier des charges ont été effectué. Il s'agit de :
\begin{itemize}
    \item Définir une grammaire UON
    \item Implémenter une intégration continue
    \item Implémenter de l'autocomplétion : Les éléments proposé à l'utilisateur sont des mots-clés : (Le nom des structures, les types et leur propriétés associés).
    \item Implémenter une outline view.
    \item Implémenter la documentation on hover
\end{itemize}

% • perspectives et recommandations
Les perspectives concernant ce sujet sont vastes, mais des améliorations possibles sont les suivants :
\begin{itemize}
    \item Implémenter les points du CDC dans la partie "si le temps le permet"
    \item Créer un langage server et déplacer, adapter le code pour pouvoir l'utiliser.
    \item Continuer à améliorer la grammaire et adapté les fonctionnalités en conséquence
\end{itemize}

Étant donné l'étendu d'un tel sujet, le travail s'est focalisé sur une infime partie de ce qui est possible de faire.
Une extension pouvant être étendue à l'infini, chacun pourrait donc ajouter ce qu'il trouve pertinent.

% TODO : English
