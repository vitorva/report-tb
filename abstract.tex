% Francais

% • le contexte,
En 2018, le professeur Yves Chevallier a imaginé un nouveau format de sérialisation proche de YAML et JSON nommé \Gls{uon}.

UON vise à rassembler toutes les caractéristiques utiles des formats de sérialisation les plus utilisés sur internet (XML, YAML et JSON),
en un seul format qui les englobe. Ce qui le rend adapté à la communication \Gls{m2m} pour des dispositifs embarqués de faibles puissances, jusqu'aux plateformes haut de gamme basées sur le cloud.

Il offre donc des fonctionnalités supplémentaires utiles pour accroître l'interopérabilité entre différents types d'appareils. Ses principaux avantages par rapport à d'autres langages similaires sont les suivants :

\begin{itemize}
    \item Typage fort
    \item Schéma de validation
    \item Forme canonique
    \item Quantités et unités physiques
    \item Références
    \item API standard
    \item Propriétés des types
    \item Types d'utilisateurs
\end{itemize}

% • la problématique,
Pour être utilisable, ce format de sérialisation doit disposer de son propre support de langage afin qu'il soit utilisé par un plus grand nombre.

% • les objectifs du travail,
% editeur de texte ou de code ?
Ce Travail de bachelor à l'objectif de répondre à ce besoin en permettant l'utilisation du langage UON dans un éditeur de code.
Sa syntaxe devra être prise en charge à terme par les principaux éditeurs de code. Ce travail se focalisera uniquement sur VS Code.
Le support de langage consistera donc à fournir à l'utilisateur, les outils pour lui permettre la rédaction la plus optimale d'un fichier UON.

% • la méthodologie utilisée,
Les premières semaines ont été majoritairement consacrées à se renseigner.
Puis un \Gls{PoC} a été rapidement réalisé pour s'assurer de la faisabilité de certaines fonctionnalités, avant de se consacrer à la réalisation du projet.
Ce PoC testait surtout que le moteur de complétion retournait bien des candidats pour la fonctonnalité de l'auto-complétion.

Puis les fonctionnalités ont été implémenté progressivement.

Elles ont connu des améliorations durant le long du projet.
Le support de langage traitait initialement uniquement une grammaire en UON qui permettait des constructions dans un format proche en JSON.
avant d'ajouter le format YAML.
Les propriétés n'étaient pas aussi considérées dans un premier temps dans l'outline view. Puis elles ont été rajoutées par la suite.

Les principales ressources utilisées ont été :

\begin{itemize}
    \item Le site de VS Code comme source principale pour se renseigner sur la mise en place et la gestion d'une extension.
    \item Des articles en ligne concernant ANTLR et le moteur de complétion.
    \item Les issues sur Github.
    \item Et bien évidemment stackoverflow pour tous types de questions.
\end{itemize}

%méthodologie
%odre
%comment gérer le panning
%ordonner tache
%reagir a un problme
%site consulter
%premier long travail de longe allein
%comment j'ai réagie face à ça

% • les principaux résultats de la recherche ou, si confidentiel, au moins la nature
%   des résultats,
Au terme de ce projet, les points attendus du cahier des charges ont été effectués. Il s'agit de :
\begin{itemize}
    \item Définir la grammaire du langage UON utilisée.
          \subitem Elle fournit du support de langage sur une grammaire fonctionnelle UON qui est capable de traiter des mappings et des séquences de type JSON et YAML, ainsi qu'un schéma de validation.
    \item Implémenter une intégration continue.
    \item Implémenter de l'autocomplétion
          \subitem les éléments proposés à l'utilisateur sont des mots-clés (le nom des structures, des unités et les types et leurs propriétés associées).
    \item Implémenter une outline view.
    \item Implémenter l'affichage des informations au survol de la souris (documentation on hover).
    \item Commencer à implémenter un Linter simple pour signaler des erreurs.
\end{itemize}

% • perspectives et recommandations
Les perspectives concernant ce sujet sont vastes, mais des améliorations possibles à ce projet sont les suivants :
\begin{itemize}
    \item Implémenter les points du CDC dans la partie "si le temps le permet"
    \item Créer un langage server et déplacer, adapter le code pour pouvoir l'utiliser.
    \item Continuer à améliorer la grammaire et adapter les fonctionnalités en conséquence.
\end{itemize}

Étant donné l'étendue d'un tel sujet, le travail s'est focalisé sur une partie de ce qu'il est possible de faire.
Une extension pouvant être étendue selon les besoins, chacun pourrait ajouter ce qu'il trouve pertinent pour améliorer le support du langage UON.

