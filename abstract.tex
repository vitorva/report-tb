% Francais

% • le contexte,
En 2018, le professeur Yves Chevallier a imaginé un nouveau format de sérialisation proche de YAML et JSON nommé \Gls{uon}.
UON vise à rassembler toutes les caractéristiques utiles des formats de sérialisation les plus utilisés sur internet (XML, YAML et JSON),
en un seul format qui les englobe. Cela dans le but de le rendre adapté à la communication \Gls{m2m} pour des dispositifs embarqués de faibles puissances, jusqu'aux plateformes haut de gamme basées sur le cloud.

% • la problématique,
Ce Travail de Bachelor a pour objectif de permettre l'utilisation du langage UON dans l'éditeur de code VS Code, en créant une extension disponible depuis le Marketplace de Visual Studio Code.
Cette extension doit fournir à l'utilisateur, le support de langage permettant une meilleure rédaction d'un fichier UON.

Le support est fourni sur une implémentation de la grammaire issue de la spécification UON.
L'API de VS Code est directement contactée pour implémenter les fonctionnalités.
ANTLR est le générateur de parser qui a été choisi.
Le moteur de complétion antlr4-c3 est utilisé comme source principale des suggestions pour l'auto-complétion.

Au terme de ce projet, les points attendus du cahier des charges ont été effectués. Il s'agit de :
\begin{itemize}
    \item Disposer d'une grammaire du langage UON utilisable
    \item Implémenter une intégration continue
    \item Implémenter une coloration syntaxique
    \item Implémenter de l'auto-complétion
    \item Implémenter une outline view
    \item Implémenter l'affichage des informations au survol de la souris (Hover Information)
    \item Implémenter un Linter simple pour signaler des erreurs
\end{itemize}

\vspace{\parskip}

% • perspectives et recommandations
Les perspectives concernant ce sujet sont vastes, mais des améliorations possibles à ce projet sont les suivantes :
\begin{itemize}
    \item Implémenter les points du CDC dans la partie "si le temps le permet".
    \item Utiliser un langage server au lieu de l'API VS Code.
    \item Continuer à améliorer la grammaire et adapter les fonctionnalités en conséquence.
\end{itemize}