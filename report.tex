\documentclass[
    iai, % Saisir le nom de l'institut rattaché
    eai, % Saisir le nom de l'orientation
    %confidential, % Décommentez si le travail est confidentiel
]{heig-tb}

\usepackage[nooldvoltagedirection,european,americaninductors]{circuitikz}

\signature{mbernasconi.svg}

\makenomenclature
\makenoidxglossaries
\makeindex

\addbibresource{bibliography.bib}

\usepackage{etoolbox}
\renewcommand\nomgroup[1]{%
  \item[\bfseries
  \ifstrequal{#1}{A}{Constantes physiques}{%
  \ifstrequal{#1}{B}{Groupes}{%
  \ifstrequal{#1}{C}{Autres Symboles}{}}}%
]}

\newcommand{\nomunit}[1]{%
\renewcommand{\nomentryend}{\hspace*{\fill}#1}}

\nomenclature[A, 02]{\(c\)}{\href{https://physics.nist.gov/cgi-bin/cuu/Value?c}
{Vitesse de la lumière dans le vide}
\nomunit{\SI{299792458}{\meter\per\second}}}

\nomenclature[A, 03]{\(h\)}{\href{https://physics.nist.gov/cgi-bin/cuu/Value?h}
{Constante de Planck}
\nomunit{\SI[group-digits=false]{6.62607015e-34}{\joule\per\hertz}}}

\nomenclature[A, 01]{\(G\)}{\href{https://physics.nist.gov/cgi-bin/cuu/Value?bg}
{Constante de gravitation universelle}
\nomunit{\SI[group-digits=false]{6.67430e-11}{\meter\cubed\per\kilogram\per\second\squared}}}

\nomenclature[B, 03]{\(\mathbb{R}\)}{Nombres réels}
\nomenclature[B, 02]{\(\mathbb{C}\)}{Nombres complexes}
\nomenclature[B, 01]{\(\mathbb{H}\)}{Quaternions}

\nomenclature[C]{\(V\)}{Volume constant}
\nomenclature[C]{\(\rho\)}{Indice de frottement sec}

\newacronym{gcd}{GCD}{Plus grand diviseur commun}
\newacronym{lcm}{LCM}{Plus petit multiple commun}
\newacronym{uon}{UON}{Unified Object Notation}
\newacronym{antlr}{ANTLR}{Unified Object Notation}
\newacronym{ebnf}{EBNF}{Extended Backus–Naur Form}

\newglossaryentry{heig-vd}{
    name=HEIG-VD,
    description={Haute École d'Ingénierie et de Gestion du canton de Vaud}
}
\newglossaryentry{hes-so}{
    name=HES-SO,
    description={Haute École Supérieure de Suisse Occidentale}
}
\newglossaryentry{latex}{
    name=latex,
    description={Un langage et un système de composition de documents}
}
\newglossaryentry{maths}{
    name=mathematics,
    description={Les mathematiques sont ce que les mathématiciens fonts}
}
\newglossaryentry{token}{
    name=token,
    description={C'est un segment de texte avec un type associé}
}
\newglossaryentry{grammaire}{
    name=grammaire,
    description={un fichier décrivant formellement un langage}
}
% Auteur du document (étudiant-e) en projet de Bachelor
\author{Vitor Vaz Afonso}

% Activer l'option pour l'accord du féminin dans le texte
\genre{male}

% Titre de votre travail de Bachelor
\title{Support du langage UON sous VS Code}

% Le sous titre est optionnel
\subtitle{Travail de Bachelor}

% Nom du professeur responsable
\teacher {Prof. Y. Chevallier (HEIG-VD)}

% Mettre à jour avec la date de rendu du travail
\date{\today}

% Numéro de TB
\thesis{7212}



\surroundwithmdframed{minted}

%% Début du document
\begin{document}
\selectlanguage{french}
\maketitle
\frontmatter
\clearemptydoublepage

%% Requis par les dispositions générales des travaux de Bachelor
\preamble
\authentification

%% Résumé / Version abbrégée
\begin{abstract}
    % Francais

% • le contexte,
En 2018, le professeur Yves Chevallier a imaginé un nouveau format de sérialisation proche de YAML et JSON nommé \Gls{uon}.
UON vise à rassembler toutes les caractéristiques utiles des formats de sérialisation les plus utilisés sur internet (XML, YAML et JSON),
en un seul format qui les englobe. Cela dans le but de le rendre adapté à la communication \Gls{m2m} pour des dispositifs embarqués de faibles puissances, jusqu'aux plateformes haut de gamme basées sur le cloud.

% • la problématique,
Ce Travail de Bachelor a pour objectif de permettre l'utilisation du langage UON dans l'éditeur de code VS Code, en créant une extension disponible depuis le Marketplace de Visual Studio Code.
Cette extension doit fournir à l'utilisateur, le support de langage permettant une meilleure rédaction d'un fichier UON.

Le support est fourni sur une implémentation de la grammaire issue de la spécification UON.
L'API de VS Code est directement contactée pour implémenter les fonctionnalités.
ANTLR est le générateur de parser qui a été choisi.
Le moteur de complétion antlr4-c3 est utilisé comme source principale des suggestions pour l'auto-complétion.

Au terme de ce projet, les points attendus du cahier des charges ont été effectués. Il s'agit de :
\begin{itemize}
    \item Disposer d'une grammaire du langage UON utilisable
    \item Implémenter une intégration continue
    \item Implémenter une coloration syntaxique
    \item Implémenter de l'auto-complétion
    \item Implémenter une outline view
    \item Implémenter l'affichage des informations au survol de la souris (Hover Information)
    \item Implémenter un Linter simple pour signaler des erreurs
\end{itemize}

% • perspectives et recommandations
Les perspectives concernant ce sujet sont vastes, mais des améliorations possibles à ce projet sont les suivants :
\begin{itemize}
    \item Implémenter les points du CDC dans la partie "si le temps le permet".
    \item Utiliser un langage server au lieu de l'API VS Code.
    \item Continuer à améliorer la grammaire et adapter les fonctionnalités en conséquence.
\end{itemize}
\end{abstract}

%% Sommaire et tables
\clearemptydoublepage
{
    \tableofcontents
    \let\cleardoublepage\clearpage
    \listoffigures
    \let\cleardoublepage\clearpage
    \listoftables
    \let\cleardoublepage\clearpage
    \listoflistings
}

\printnomenclature
\clearemptydoublepage
\pagenumbering{arabic}

%% Contenu
\mainmatter
\chapter{Introduction}
L'introduction est une section requise dans un rapport technique. Introduisez votre travail, l'idée de départ et les objectifs attendus. Un lecteur qui découvrirait votre projet au travers de cette introduction devrait ainsi être capable d'en comprendre le cadre, l'idée générale et les aboutissants du projet.

\section{Contexte}
Cette section \underline{n'est pas obligatoire}, mais elle est souvent présente dans un rapport technique pour compléter l'introduction et définir le contexte du travail \cad le cadre formel dans lequel le travail est mené.

%%if
\section{Citations et bibliographie}
Citer vos sources est essentiel. Avec \texttt{biblatex} vous pouvez facilement citer des articles, des livres ou des sites internet. Toutes les citations dans le texte seront automatiquement regroupées en fin de document dans la section \guillemotleft Bibliographie\guillemotright. Par exemple, citons un article d'Einstein \cite{einstein} ou le livre de Dirac \cite{dirac}.

Parfois il peut être utile d'utiliser un gestionnaire de bibliographie. La communauté académique recommande l'outil \href{https://www.zotero.org/}{Zotero} qui permet de gérer une bibliothèque numérique d'ouvrages et de références numériques. Il permet également de générer une bibliographie compatible avec \LaTeX.

\section{Exemple d'équation}
L'une des principales forces de \LaTeX est la saisie d'équations. L'équation \ref{eq:1}, citée à titre d'exemple, représente la transformation de phase d'une lentille biconvexe. Pour rédiger une équation \LaTeX vous pouvez utiliser des outils en ligne tels que \href{https://www.latex4technics.com/}{latex4technics}.

\begin{equation} \label{eq:1}
    \begin{split}
        L(x,y) &= \exp\left( - i\frac{{2\pi }}{\lambda }\left( {n\Delta \varphi (x,y) + \Delta {\varphi _0} - \Delta \varphi (x,y)} \right)\right)\\
        &= {\exp\left({i\frac{{2\pi }}{\lambda }\Delta {\varphi _0}}\right)}{\exp\left({ - i\frac{{2\pi }}{{\lambda f}}({x^2} + {y^2})}\right)}
    \end{split}
\end{equation}

\section{Exemples de diagrammes}

Les diagrammes de flux peuvent être réalisés en utilisant l'outil \href{https://app.diagrams.net/}{draw.io}. Une exportation en \texttt{.xml} (non compressé) permet de garder les sources de la figure. Le rendu en \texttt{.pdf} sera réalisé à la volée à la compilation. L'intérêt est double : n'avoir qu'une source de vérité \cad pas d'image intermédiaire à stocker, et réduire la quantité d'information stockée.

Puisque la source est au format XML, les textes sont accessibles au correcteur orthographique et il vous est rendu possible les modifier sans avoir à éditer l'image. La figure \ref{euclide.xml} en est un exemple.


\figi{euclide.xml}{9cm}{Algorithme d'Euclide}

Notons qu'il est inutile d'insérer des images coloriées là où la couleur n'offre aucune valeur ajoutée ; évitez également les ombrages et autres effets de style. Enfin, préférez toujours des représentations vectorielles là où c'est possible.

Voici un autre type de diagramme utile (figure \ref{sequence.xml}), celui d'une séquence UML.

\figi{sequence.xml}{8cm}{Diagramme de séquence}

\section{Exemple de figure}

Pour présenter des résultats d'expérience, vous pouvez soit dessiner des graphiques manuellement en utilisant des outils de dessin vectoriel comme Inkscape ou Adobe Illustrator comme illustré à la figure \ref{plot.svg} ou alors, vous pouvez utiliser Python ou Matlab. Avec ce dernier choix, vous pouvez générer vos figures à la volée : le code source \ref{python} permet de générer la figure \ref{bode.py}.

\fig{plot.svg}{Exemple de graphique plan}

\begin{listing}[h]
    \inputminted{python}{assets/figures/bode.py}
    \caption{Génération d'un diagramme de Bode \label{python}}
\end{listing}

\figi{bode.py}{12cm}{Diagramme de Bode généré à la volée}

\clearpage

\subsection{Schémas électroniques}
Vous pouvez également utiliser TikZ pour créer vos propres schémas électriques et électroniques comme l'exemple \ref{circuit}.

\begin{figure}[h]
    \begin{center}
        \begin{circuitikz}
            \draw
            (0,0) to [short, *-] (6,0)
            to [V, l_=$\mathrm{j}{\omega}_m \underline{\phi}^s_R$] (6,2)
            to [R, l_=$R_R$] (6,4)
            to [short, i_=$\underline{i}^s_R$] (5,4)
            (0,0) to [open, v^>=$\underline{u}^s_s$] (0,4)
            to [short, *- ,i=$\underline{i}^s_s$] (1,4)
            to [R, l=$R_s$] (3,4)
            to [L, l=$L_{\sigma}$] (5,4)
            to [short, i_=$\underline{i}^s_M$] (5,3)
            to [L, l_=$L_M$] (5,0);
        \end{circuitikz}
        \caption{Circuit électrique \label{circuit}}
    \end{center}
\end{figure}

\subsection{Dessins techniques}
L'intégration de dessins mécaniques est préférée en vue filaire. SolidWorks conserve la représentation vectorielle à l'exportation. À partir du PDF généré, l'image peut être isolée et sauvegardée en format SVG.

\begin{figure}[!ht]
    \begin{center}
        \includegraphics[width=10cm]{\assetsdir/assembly.svg.\graphicsExt}
    \end{center}
    \caption[Assemblage mécanique]{\label{assembly}Réducteur cycloïdale de puissance comportant 6. l'axe de sortie, 14. le roulement de sortie, 1. le corps du réducteur en aluminium, 3 et 5. les disques cycloïdaux et 2. les goupilles de prise... D'autres informations liées à la figure elle-même peuvent aussi figurer dans la légende}
\end{figure}

Notez ici que la légende est particulièrement longue. Celle que vous retrouverez dans la table figures est plus courte. La commande \mintinline{latex}{\caption[courte]{longue}} permet de saisir une légende courte, pour la table des figures et longue pour le corps du document.

La figure \ref{assembly} est un dessin technique épuré qui permet de décrire un phénomène ou un fonctionnement important dans le rapport technique. Les mises en plan détaillées seront quant à elles disponibles en annexes.

\clearpage
\section{Tableaux}

Concernant les tableaux, restez simple et minimaliste, n'ajoutez des séparateurs que là ou c'est nécessaire pour améliorer la lisibilité. Une liste de quelques cantons suisses est donnée à titre d'exemple dans la table \ref{cantons}.

\begin{table}[h]
    \begin{center}
        \caption{Liste des cantons \label{cantons}}
        \begin{tabular}{c|l|r}
            Abréviation & Nom du canton & Depuis                  \\ \hline
            ZH          & Zürich        & \ordinalnum{1} mai 1351 \\
            BE          & Berne         & 6 mars 1353             \\
            FR          & Fribourg      & 22 décembre 1481        \\
            VD          & Vaud          & 19 février 1815         \\
            VS          & Valais        & 4 août 1815             \\
            NE          & Neuchâtel     & 19 mai 1815             \\
            GE          & Genève        & 19 mai 1815
        \end{tabular}
    \end{center}
\end{table}

Si vous devez donner une spécification technique, n'oubliez pas de mentionner les valeurs minimales, maximales et nominales sans omettre l'unité de mesure. Notez que les séparateurs verticaux sont souvent critiqués pour réduire la lisibilité mais parfois ils sont utiles. Utilisez-les avec parcimonie.

\begin{table}[h]
    \begin{center}
        \caption{Exigences techniques \label{specification}}
        \begin{tabularx}{\textwidth}{cXcccc}
            No. & Exigence                                                                   & Min. & Nom. & Max. & Unité                           \\ \toprule
            E1  & Tension d'alimentation                                                     & 12   & 24   & 48   & \si{\volt}                      \\ \midrule
            E2  & Fréquence                                                                  & 50   &      & 60   & \si{\hertz}                     \\ \midrule
            E3  & Concentration                                                              &      & 300  & 1200 & \si{\nano\gram\per\milli\litre} \\ \midrule
            E4  & \multicolumn{5}{l}{Doit pouvoir être stoppé à l'aide d'un arrêt d'urgence}
        \end{tabularx}
    \end{center}
\end{table}

L'exemple de la table \ref{specification}, assigne pour chaque exigence un numéro unique. Cette table est \textbf{normative}, chaque élément doit pouvoir être référencé par un identifiant unique (cf. T\ref{specification}-E3). Dans le cas ou cet identifiant est utilisé en dehors de ce document, la version du document devra être renseignée.

\section{Index}
\LaTeX~ permet d'indexer les mots \index{mots} importants. Il suffit de placer les termes importants d'un paragraphe dans la commande \texttt{\textbackslash index\{terme\}} et ils apparaîtront automatiquement à la fin de ce rapport dans l'index du document.

\index{Napoléon}

Imaginons que dans cette section nous parlions du cheval blanc \index{cheval blanc} de Napoléon. Il se pourrait que le lecteur recherche ce passage dans la version imprimée du rapport. Avec l'index, rien de plus facile. Allez jeter un oeil à la page \pageref{index}.

\section{Notes de bas de page}

\maraja{Je suis une marginale, et je suis utile pour résumé un paragraphe en quelques mots.} Parfois, il est plus élégant d'annoter une définition en utilisant une note de bas de page \footnote{La note en bas de page (ou note de bas de page) est une forme littéraire, consistant en une ou plusieurs lignes ne figurant pas dans le texte.}. Alternativement il est possible d'annoter un paragraphe avec une note marginale.

\section{Glossaire et acronymes}

La \Gls{heig-vd} membre de la \Gls{hes-so} propose ce modèle de document. Le format \LaTeX est particulièrement adapté pour les documents qui contiennent des expressions mathématiques. Pour plus de détail sur l'utilisation d'un glossaire, se référer à \url{https://www.overleaf.com/learn/latex/Glossaries}. Tient donc, ci-dessus nous utilisons deux acronymes. Les trouverez-vous dans le glossaire en page \pageref{glossaire} ?

\section{Unités de mesure}

Lorsque vous mentionnez des quantités, utilisez les unités du système international. \LaTeX~et le paquet \textsf{siunitx} permet la saisie de quantités. La commande suivante permet d'afficher \SI{42.12}{\kilo\gram\metre\per\square\second}.\par

\mintinline{latex}{\SI{42.12}{\kilo\gram\metre\per\square\second}}\par
%%fi

\chapter{Conclusion}

%%if
Bien que non nécessaire dans un rapport de Bachelor, la discussion finale d'un projet résume les résultats obtenus et dresse une conclusion objective du projet. Un manager de société est souvent amené à lire de nombreux rapport, il ne s'intéresse généralement qu'à l'introduction au contexte de l'étude et à sa conclusion.

Il est de coutume de signer la conclusion...
%%fi

\vfil
\hspace{8cm}\makeatletter\@author\makeatother\par
\hspace{8cm}\begin{minipage}{5cm}
%%if
    % Place pour signature numérique
    \printsignature
%%fi
\end{minipage}
\clearpage

\appendix
\appendixpage
\addappheadtotoc

%%if
\chapter{Première annexe}

Les annexes n'ont pas un contenu \underline{normatif} mais \underline{descriptif}. Tout contenu annexé ne doit pas être nécessaire à la bonne compréhension du travail.

Les annexes contiennent généralement :

\begin{itemize}
    \item les dessins mécaniques (mises en plan);
    \item les schémas électriques détaillés;
    \item des photographies du projet;
    \item des scripts et des extraits de code source;
    \item des documents techniques \pex \emph{datasheet};
    \item des développements mathématiques.
\end{itemize}
\section{Sous section}
\lipsum[1]
%%fi

\let\cleardoublepage\clearpage
\backmatter

\label{glossaire}
\printnoidxglossary
\printbibliography
\label{index}
\printindex

%%if
\clearpage
\Large\textbf{Colophon :}\par\normalsize
\thispagestyle{empty}
La qualité de cet ouvrage repose que le moteur \LaTeX. La mise en page et le format sont inspirés d'ouvrages scientifiques tels que le modèle de thèse de l'EPFL et celui des publications O'Reilly.

Les diagrammes et les illustrations sont édités depuis l'outil en ligne draw.io. Certaines illustrations ont été reprises dans Adobe Illustrator. Les représentations 3D sont exportées de SolidWorks et certains graphiques sont générés à la volée depuis un code source Python.

L'auteur fictive de ce document \emph{Maria Bernasconi} est un nom emprunté, par amusement, aux spécimens publiés par Postfinance.

Ce document a été compilé avec XeLaTeX.

La famille de police de caractères utilisée est \emph{Computed Modern} créée par Donald Knuth avec son logiciel METAFONT.
\vfil
Le Colophon est le dernier élément d'un document qui contient des notes de l'auteur concernant la mise en page et l'édition du document : il est parfaitement optionnel.
%%fi

\end{document}
